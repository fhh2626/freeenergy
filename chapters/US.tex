% !TeX spellcheck = en_US
% !TeX encoding = UTF-8
\section{Umbrella Sampling\label{Sec:ES:US}}
Umbrella Sampling method was proposed by Torrie and Valleau in 1977,\cite{TorrieJComputP1977} and is still widely used nowadays.
Suppose we are studying a transition process between two states such as conversion between two dominant conformations or a chemical reaction, and these two states are separated by a high barrier relative to $k_BT$. Therefore, the transition is a rare event. A schematic representation of the free energy landscape is shown in Fig.~\ref{Fig:ES:dual_harmonic}.
\begin{figure}[htbp]
	\centering
%	\resizebox{2cm}{!}{
		\begin{tikzpicture}
		\def\lims{xmin=-5,xmax=5,ymin=-30,ymax=14}
		\begin{axis}[\lims,hide x axis, hide y axis,width=0.65\textwidth,height=0.4\textwidth]
           \addplot[mark=none,smooth,domain=-5:5] {0.4*((x+2)*(x-2))*((x+2)*(x-2))-4*x*x};
		\end{axis}
		\end{tikzpicture}
%	}
	\caption{A typical free energy surface. Two free energy wells are separated by a barrier higher than $k_BT$.}\label{Fig:ES:dual_harmonic}
\end{figure}

%\begin{figure}[htbp]
%	\centering
%	\includegraphics[width=0.6\textwidth]{figures/dual_harmonic.pdf}\\
%	\caption{A typical free energy surface. Two free energy barriers are separated by a barrier higher than $kT$.}\label{Fig:ES:dual_harmonic}
%\end{figure}

Sometimes, we are interested in not only these two dominant states but also the whole pathway. Usually, we define a reaction coordinate $\xi$, including one or more collective variables (CVs), which can distinguish the two minima and characterize the free-energy pathway. Then we want to know the free-energy change along the reaction coordinate, $\xi$. It should be noted that choosing a suitable set of CVs is nontrivial for most cases.\cite{LeitoldJCP2020} A CV can be either a real coordinate such as the difference of bond lengths in, for example, an $S_N2$ reaction, or a thermodynamics coupling parameter ($\lambda$) that defines an unphysical path. If we run a simulation with the reaction coordinate set to a local maximum, i.e. the system being the transition state, the system will quickly roll back to the ``reactant'' or the ``product'' state in order to reduce the free energy. The consequence is that phase space outside the ``reactant'' and ``product'' regions cannot be sampled sufficiently to yield accurate free energy profile in a brute force simulation. In order to enhance the exploration in these regions, we can added a bias potential $\Delta V(\xi)$ into the system, guaranteeing
\begin{equation}
	\forall \xi_{1} \; \mathrm{and} \; \xi_{2}, \left [ U(\xi_{1}) + \Delta V(\xi_{1}) \right ]  - \left [ U(\xi_{2}) + \Delta V(\xi_{2}) \right ] < k_BT
\end{equation}
then the free-energy surface can be explored within the timescale amenable to MD simulations. $\Delta V(\xi)$ is called the ``umbrella potential''.

However, as the free-energy surface is usually not known \textit{a priori}, it is difficult to determine $\Delta V(\xi)$. To circumvent this issue, we can stratify the free-energy pathway into multiple \textit{windows}, namely break up the reaction-coordinate space into ``parts'', by a series of (usually harmonic) restraints, $\Delta U_i(\xi)$. In other words, the $i$th simulation, or the simulation of the $i$th window, is performed on the potential energy surface 
\begin{equation}
	U_i(\mathbf{R})=U_0(\mathbf{R})+\Delta U_i(\xi).
\end{equation}
The strengths of the biases should be strong enough to maintain the system in the vicinity of where you are interested in, and also should be weak enough that the system can have significant overlap between two adjacent windows. At the same time, the windows should be small enough to guarantee in each window,
\begin{equation}
	\forall \xi_{1} \; \mathrm{and} \; \xi_{2}, \left [ U(\xi_{1}) + \Delta U_i(\xi_{1}) \right ]  - \left [ U(\xi_{2}) + \Delta U_i(\xi_{2}) \right ] < k_BT
\end{equation}
After all the simulations, the (biased) distribution of the samples in the whole region should be as flat as possible. Ensemble average under $U_0$ can be calculated from the ensembles generated under the biased Hamiltonians $U$ via
\begin{align}
	\left<X(\mathbf{R})\right>_0=&\frac{\int X(\mathbf{R})\exp{\left[-\beta U_0(\mathbf{R})\right]}\diff\mathbf{R}}{\int \exp{\left[-\beta U_0(\mathbf{R})\right]}\diff\mathbf{R}}\notag\\
	                            =&\frac{\int X(\mathbf{R})\exp{\left[\beta \Delta U_i(\mathbf{R})\right]}\exp{\left[-\beta U_i(\mathbf{R})\right]}\diff\mathbf{R}}{\int \exp{\left[\beta \Delta U_i(\mathbf{R})\right]}\exp{\left[-\beta U_i(\mathbf{R})\right]}\diff\mathbf{R}}\notag\\
	                =&\frac{\left<X\exp{\left(\beta\Delta U_i\right)}\right>_i}{\left<\exp{\left(\beta\Delta U_i\right)}\right>_i}.
	\label{Eq:ES:US:reweighting}
\end{align}
Better postprocessing methods are the Weighted Histogram Analysis Method, Umbrella Integration and the Multistate Bennett Acceptance Ratio method (to be discussed in Section~\ref{Sec:FEM:WHAM}, ~\ref{Sec:FEM:UI} and~\ref{Sec:FEM:MBAR}).